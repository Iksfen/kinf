\documentclass[a4paper, pdftex]{amsart}

\usepackage[latin2]{inputenc}
\usepackage[T1]{fontenc}
\usepackage{url}

\begin{document}

\thispagestyle{empty}
\noindent{\Large K�ko Informatyczne, 10 lutego 2011}

\vspace{5mm}\noindent
Zaczniemy od nast�puj�cej zawarto�ci pliku {\tt .js}:

\begin{verbatim}

var nic  = "black";
var ziemia = "brown";
var kamien = "gray";

var stany = Array( nic, ziemia, kamien );
                                        
function setupBoard()
{        
}
                        
function calculateNewState(x,y)
{
        var teraz = cellState(x,y);
        var pozniej;

        // tu bedziemy wpisywac kod

        return pozniej;
}
\end{verbatim}

\vspace{5mm}\noindent
Zamiast przepisywa�, mo�esz skopiowa� ten kod ze strony

\vspace{2mm}
\url{http://www.mimuw.edu.pl/~amn/ki/10lutego/bd.html}

\vspace{5mm}
Po za�adowaniu pliku {\tt.html} (tego, co poprzednio) powinna pojawi� si� czarna plansza. 
  Klikaj�c na ni� powinno by� mo�na zmieni� kolory na br�zowy (ziemia) i szary (kamie�). 
  Sprawd� to!
  
\vspace{2mm}
{\bf  Nasz cel na dzi�
  to implementacja fizyki kamieni.}
 
\vspace{2mm}
Osi�gniemy to etapami:

\begin{enumerate}
\item Spadanie kamieni w d� (�atwe).
\item Toczenie si� kamieni w lewo-d� (troch� mniej �atwe).
\item Toczenie si� kamieni w prawo-d� (wcale nie �atwe).
\end{enumerate}

Nie zaczynaj implementowa� nast�pnego etapu, je�eli nie dzia�a Ci wsze�niejszy. Najpierw
 wszystko om�wimy przy tablicy.
 
\end{document}
