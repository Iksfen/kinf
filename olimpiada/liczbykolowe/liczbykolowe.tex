\documentclass[11pt,a4paper,pdftex]{amsart}

\usepackage[latin2]{inputenc}
\usepackage[T1]{fontenc}
\usepackage[polish]{babel}

\begin{document}

\thispagestyle{empty}

\noindent{\large K�ko Informatyczne, 17 stycznia 2011, $15^{30} - 17^{30}$.}
\vspace{5mm}

\noindent{\Large \bf Liczby Ko�owe}
\vspace{5mm}

{\em Ko�ow� liczb� pierwsz�} nazywamy tak� liczb� pierwsz�, kt�ra po przeniesieniu na pocz�tek dowolnej liczby cyfr z ko�ca pozostaje liczb� pierwsz�. Przyk�adowo, $1193$ jest ko�ow� liczb� pierwsz�, bo liczby $1193$, $3119$ (przeniesiona ostatnia cyfra), $9311$ (przeniesione dwie ostatnie cyfry) i $1931$ (przeniesione trzy cyfry) s� pierwsze (cho� np. $1139 = 17 \cdot 67$ nie jest pierwsza).

\vspace{5mm}
\noindent{\bf Zadanie}
\vspace{1mm}

Napisz program, kt�ry wyznaczy wszystkie liczby ko�owe $\leq 10^{9}$.

Uwaga: $10^9 < 2^{30}$.

\vspace{3mm}
Dost�pna pami��: $32$MB. Szczodry limit czasu: $10$s.

\vspace{5mm}
\noindent{\bf Zadania pomocnicze}
\vspace{1mm}

{\bf 1.} Napisz funkcj� sprawdzaj�c�, czy dana liczba $\leq 10^9$ jest pierwsza.

\vspace{1mm}
{\bf 2.} Czy mo�na przyspieszy� dzia�anie tej funkcji wiedz�c, �e b�dzie ona wywo�ywana wielokrotnie
  dla r�nych liczb $\leq 10^9$?
  
\vspace{1mm}
{\bf 3.} Napisz funkcj� przenosz�c� ostatni� cyfr� liczby na jej pocz�tek (zastan�w si�, co si� dzieje w przypadku zera).

\vspace{1mm}
{\bf 4.} Liczba $11593$ jest pierwsza. Czy jest ko�ow� liczb� pierwsz�?

\end{document}
